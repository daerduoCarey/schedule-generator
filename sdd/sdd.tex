\documentclass{article}
\usepackage[dutch]{babel}
\usepackage[official]{eurosym}
\usepackage{graphicx}
\usepackage{epic,eepic}
\usepackage{mathtools}
\usepackage{caption}
\usepackage{textcomp}
%\graphicspath{{fotosklein/}}


\title{Software Design Descriptions voor Schedule-Generator}
\author{Matthias Caenepeel \and Adam Cooman \and Alexander De Cock \and Zjef Van de Poel}
\date{23 februari 2011 Versie 0.1} 

\begin{document}

\maketitle

\newpage

% \newpage
% Signature page

% \newpage

\section*{Aanpassingsgeschiedenis}
\begin{itemize}
\item[.] 23/2/2011 versie 0.1: Aanmaak document \\[-3mm]
\end{itemize}

%\section*{Preface}

\newpage
\tableofcontents

\newpage
\section{Introduction}
\subsection{Purpose}
\subsection{Scope}
\subsection{Context}
\subsection{Summary}

\newpage
\section{References}

\newpage
\section{Glossary}

\newpage
\section{Body}
\subsection{Identified stakeholders and design concerns}

\subsection{Design viewpoint 1: Compositie}
% samenwerking van alle onderdelen + uitleg
% Alexander schrijft dit
\subsubsection{Design concerns}
Alle verschillende onderdelen opsommen en hun taken beschrijven.

\subsubsection{Design elements}
Entiteiten, verbanden en attribuut.

\subsubsection{Function attributes}
Legt het verband tussen de entiteiten. (da's eigenlijk de transferfunctie ma Rahnild weet toch wa da is; dus laat zitten jong)

\subsubsection{Subordinates attributes}
Entiteiten in entiteiten (overal UML als exampletaal)

\subsection{Design viewpoint 2: Logical}
% UML diagramma + uitleg
% Zjef krijgt dit
% zie ook minutes van de eerste vergadering
\subsubsection{Design concerns}
Alle verschillende onderdelen opsommen en hun taken beschrijven.
\subsubsection{Design elements}
klassen, verbanden en attribuut.


\subsection{Design viewpoint 3: Interactie tussen site en server}
% HTTP
% Adam schrijft dit het is toegelaten dat je zelf de punten em beke invult
\subsubsection{Design concerns}
Alle verschillende onderdelen opsommen en hun taken beschrijven.
\subsubsection{Design elements}
klassen, verbanden en attribuut.


\subsection{Design viewpoint 4: Algoritme voor kalenderplanning}
% Matthias schrijft dit wel
\subsubsection{Design concerns}
Alle details nodig voor de programmeur (dat ben jij) die nodig zijn om het algoritme voor te stellen/testplan om te zien of het werkt
\subsubsection{Design elements}
de data die je nodig hebt om het algoritme te laten bollen

\subsection{}



\subsection{Design rationale}


%\begin{figure}
%\begin{center}
%\includegraphics[width=\textwidth]{Bangladesh}
%\caption*{Bangladesh}
%\end{center}
%\end{figure}

%\begin{tabular}[t]{llll}
%Mandi & Katholiek & 25\% & Platte neus -- spleetogen \\
%Gohli & Moslims   & 75\% & Grote ogen -- scherpere neus \\
%Kooch & Hindu     & 1\%  & Mengeling van de twee \\
%\end{tabular}
%\\[5mm]

%\begin{itemize}
%\item[.] Slaapmatje \\[-3mm]
%\item[.] Kleren: 3 korte broeken, 6 onderbroeken, 4 T-shirts\\[-3mm]
%\item[.] Longi: voor 's avonds en als pyjama\\[-3mm]
%\item[.] Tandenborstel en twee tubes tandpasta\\[-3mm]
%\end{itemize}


 \end{document}
