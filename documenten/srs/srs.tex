\documentclass{article}
\usepackage[dutch]{babel}
\usepackage[official]{eurosym}
%\usepackage{graphicx}
%\usepackage{epic,eepic}
\usepackage{mathtools}
\usepackage{caption}
\usepackage{textcomp}
\title{Software Requirements Specifications voor Schedule-Generator}
\author{Matthias Caenepeel \and Adam Cooman \and Alexander De Cock \and Zjef Van de Poel}
\date{23 februari 2011 Versie 1.0} 

\begin{document}

\maketitle

\newpage

% \newpage
% Signature page

% \newpage

\section*{Aanpassingsgeschiedenis}
\begin{itemize}
\item[.] 23/2/2011 versie 0.1: Aanmaak document, toevoeging  \\[-3mm]
\item[.] 25/2/2011 versie 0.2: Toevoeging tekst vanaf Performance Requirements tot einde \\[-3mm]
\item[.] 28/2/2011 versie 1.0: Toevoeging functionele vereisten, verbeteringen doorgevoerd \\[-3mm]
\end{itemize}

%\section*{Preface}

\newpage
\tableofcontents

\newpage

\section{Introduction}
\subsection{Purpose}
% Delineate the purpose of the SRS.
% Specify the intended audience for the SRS.

Het doel van de SRS is om een overzicht te geven van alle functionaliteiten die er moeten voorzien worden. \\
Het doelpubliek is het team dat aan het project werkt en de docent die het team begeleidt en evalueert. \\

\subsection{Scope}
% Identify the software product(s) to be produced by name (e.g., Host DBMS, Report Generator, etc.);
% Explain what the software product(s) will, and, if necessary, will not do;
% Describe the application of the software being specified, including relevant benefits, objectives, and goals;
% Be consistent with similar statements in higher-level specifications (e.g., the system requirements speci�cation), if they exist.

\begin{description}
\item[Schedule generator] deze software (geschreven in Java) zal in staat zijn om een lessenrooster samen te stellen dat aan bepaalde voorwaarden voldoet.  \\
\item[Website] dit is de software die de site omvat waarop de gebruikers zullen werken. \\
\item[Database] deze zal de informatie over het lessenrooster bijhouden en ordenen. \\
\item[Servelets] deze software zal ervoor zorgen dat de site doet wat de gebruiker vraagt. 
\end{description}

%\subsection{Definitions, acronyms and abbreviations}
%\subsection{references}
%Provide a complete list of all documents referenced elsewhere in the SRS;
%Identify each document by title, report number (if applicable), date, and publishing organization;
%Specify the sources from which the references can be obtained.

\subsection{Overview}
% Describe what the rest of the SRS contains;
% Explain how the SRS is organized.

De rest van de SRS zal de software vereisten van de schedule generator verder uitdiepen. De verschillende gebruikers zullen gedefinieerd worden en de functionaliteiten die die gebruikers krijgen zullen opgesomd en besproken worden.

\newpage

\section{Overall description}
% \subsection{Product perspective}

% blokdiagram van programma en verschillende softwareinterfaces die we gebruiken
% MySQL, Tomcat, Java Runtime Enviroment voor generator zelf ...

% \subsubsection{System interfaces}
% \subsubsection{User interfaces}
% \subsubsection{Hardware interfaces}
% \subsubsection{Software interfaces}
% \subsubsection{Communications interfaces}
% \subsubsection{Memory}
% \subsubsection{Operations}
% \subsubsection{Site adaptation requirements}

\subsection{Product functions}
%samenvatting van de belangrijkste functionaliteiten. \\

Gebruikers krijgen een gepersonaliseerde account waarop zij hun eigen lessenrooster alsook dat van andere richtingen kunnen opvragen (en in het geval van sommige gebruikers kunnen aanpassen).\\
Het programma bevat een database waarin informatie over de verschillende vakken wordt bijgehouden (wie is de docent, hoeveel studenten zijn er ingeschreven, waar wordt het gedoceerd,...)\\
Het bevat ook een schedule generator die in staat is een lessenrooster te genereren die aan bepaalde voorwaarden voldoet.

\subsection{User characteristics}

De gebruikers zijn studenten die in staat om zijn om de verschillende lessenroosters te bekijken en docenten die in staat zijn het rooster op te vragen, maar evenwel aanpassingen te maken aan de informatie van de vakken waarvoor zij verantwoordelijk zijn.

\subsection{constraints}

Het geheel moeten worden gedraaid op de server Wilma die als besturingssysteem Linux heeft.\\
Alle code moet open source zijn.\\
Er moet geprogrammeerd worden in Java.\\
Men heeft internet nodig om aan de diensten te kunnen.

% \subsection{Assumptions and dependencies}

\newpage

\section{Specific requirements}

\subsection{External interface requirements}

\subsubsection{User interfaces}

De opdracht gebiedt ons om met een website als user interface te werken. Om dit te verwezenlijken zal XHTML in combinatie met CSS gebruikt worden (mogelijk ook javascript). 
Na het inloggen krijgt de gebruiker een pagina te zien met verschillende tabs. Voor elke gebruikersklasse en de functionaliteiten die ze krijgen bestaat een ander tabblad. Op die manier is het gemakkelijker om functionaliteiten toe te voegen en overzichtelijk weer te geven. Guests krijgen bijvoorbeeld maar een enkele tab te zien waarin ze een naam kunnen intypen en waarin de kalender weergegeven wordt. Studenten krijgen een tweede tab erbij waarin ze hun vakkenlijst kunnen aanpassen etc.

\subsubsection{Communications and software interfaces}

Om de communicatie tussen de browser van de gebruiker en de server te doen wordt gebruik gemaakt van HTTP. De HTTP paketten zullen op de server ge\"{i}nterpreteerd worden door servlets die op de server uitgevoerd worden via Tomcat. De servlets zijn in JAVA geschreven en genereren de gevraagde XHTML en CSS code die dat terug naar de gebruiker gestuurd worden.\\ 

De servlets communiceren met de database via MySQL en interpreteren de gegevens voor de gebruiker. Het proces kan overzichtelijk weergegeven worden in volgend blokschema:

\unitlength=0.3\textwidth
\begin{center}
\begin{picture}(3,1)
%serverlijnen
\put(0.2,0.2){Server}
\put(0,0){\line(1,0){1.6}}
\put(0,0){\line(0,1){1}}
\put(1.6,0){\line(0,1){1}}
\put(0,1){\line(1,0){1.6}}
%browser
\put(2.4,0.6){Browser}
\put(2.3,0.9){\line(1,0){0.5}}
\put(2.8,0.4){\line(0,1){0.5}}
\put(2.3,0.4){\line(0,1){0.5}}
\put(2.3,0.4){\line(1,0){0.5}}
%database
\put(0.15,0.6){Database}
\put(0.1,0.9){\line(1,0){0.5}}
\put(0.6,0.4){\line(0,1){0.5}}
\put(0.1,0.4){\line(0,1){0.5}}
\put(0.1,0.4){\line(1,0){0.5}}
%servlet
\put(1.1,0.6){Servlet}
\put(1.0,0.9){\line(1,0){0.5}}
\put(1.5,0.4){\line(0,1){0.5}}
\put(1.0,0.4){\line(0,1){0.5}}
\put(1.0,0.4){\line(1,0){0.5}}
%lijn MySQL
\put(0.65,0.7){MySQL}
\put(0.62,0.65){\line(1,0){0.36}}
%lijn HTTP
\put(1.8,0.7){HTTP}
\put(1.52,0.65){\line(1,0){0.76}}
\end{picture}\\[3mm]
\caption{Overzicht van de communicatie en de gebruikte protocols tussen de verschillende delen van het programma}
\end{center}

De verschillende versies software die we gebruiken zijn de volgende:

\begin{itemize}
\item[.] MySQL: JDBC driver for MySQL 5.1.15 \\[-5mm]
\item[.] Tomcat \\[-5mm]
\end{itemize}

\subsubsection{Hardware interfaces}

De Hardware interfaces die we gebruiken worden vastgelegd door de opdrachtgever. Ons programma moet op Wilma kunnen draaien. De hardware interface die we hebben is dus een Linux besturingssysteem.

\newpage

\subsection{Functionele Vereisten}

In deze subsectie van het SRS  wordt een overzicht gegeven van alle functionaliteiten van de software. Om de ontwikkeling van de functionaliteiten tijdens het productie te kunnen opvolgen, wordt elke functionaliteit voorzien van een naam, code en omschrijving. Daarnaast worden de functionaliteiten ook nog eens opgedeeld in verschillende categorie�n en indien mogelijk toegekend aan een bepaald gebruikertype.

\subsubsection{Categorie op basis van prioriteit}
Enerzijds kan men de functionaliteiten opdelen volgens hun prioriteit. Dit leidt tot onderstaande categorie�n:
\begin{itemize}	
\item[-] \textit{noodzakelijke functionaliteiten}: functionaliteiten waarvan de aanwezigheid in het eindproduct wordt gegarandeerd omdat ze noodzakelijk zijn voor een minimale werking van het systeem. Deze functionaliteiten hebben de hoogste prioriteit tijdens de verwezenlijking van het project. \\
\item[-] \textit{mogelijke functionaliteiten}: functionaliteiten waarvan de uitvoering haalbaar is maar zonder garantie op aanwezigheid in het eindproduct. Deze functionaliteiten hebben een matige prioriteit \\
\item[-] \textit{extra functionaliteiten}: functionaliteiten waarvoor de prioriteit laag is. \\
\end{itemize}

\subsubsection{Categorie op basis van datapad}
Anderzijds kan men functionaliteiten ook opdelen volgens het pad die de data tijdens hun uitvoering volgt. Dit geeft aanleiding tot volgende categorie�n

\begin{itemize}
\item[-] \textit{Invoerfunctionaliteiten}:  alle functionaliteiten die de gebruiker toelaten informatie op te sturen naar de database op de server. 
\item[-] \textit{Uitvoerfunctionaliteiten}: alle functionaliteiten die de gebruiker toelaten informatie op te vragen uit de database op de server.
\item[-] \textit{Verwerkingsfunctionaliteiten}: alle functionaliteiten die gegevens, zonder tussenkomst van de gebruiker, uit de database verwerken en de daarbij bekomen resultaten terug schrijven naar de database op de server.
\end{itemize}

\subsubsection{Gebruikertype }
Om bepaalde functionaliteiten te kunnen plaatsen in een context wordt het gebruikertype dat toegang heeft tot deze functionaliteit indien mogelijk vermeld. Vandaar volgt hier een korte beschrijving van de gebruikertypes:
\begin{itemize}
\item[-] \textit{Software beheerder}: Verantwoordelijke voor het beheren technische aspecten van de software. Deze omvatten o.a. instaleren van de software, laten aanmaken van de databasestructuur en het configureren van de lessenroosterplanner. Daarnaast heeft deze gebruiker ook toegang tot een logboek en kan hij bepaalde gebeurtenissen ongedaan te maken.
\item[-] \textit{Account beheerder}: Deze gebruiker kan de omschrijving van bestaande gebruikertypes wijzigen en nieuwe gebruikertypes defini�ren. Hij is ook verantwoordelijk voor het aanmaken van de andere beheerder accounts.
\item[-] \textit{Rooster beheerder}: Deze gebruiker is instaat lessenrooster manueel aan te passen, lessenroosters door de roosterplanner te laten genereren en algemene beperkingen voor de roosterplanner in te stellen (bvb: feestdagen). De rooster beheerder zal ook eventuele conflict situaties in het rooster moeten oplossen. 
\item[-] \textit{Student beheerder}: Deze gebruiker is verantwoordelijk voor het aanmaken van de studenten accounts en het toekennen van programma�s en vakken aan studenten. De vakken toegewezen door de student beheerder zullen in rekening worden gebracht tijdens het opstellen van de lessenroosters. 
\item[-] \textit{Docent beheerder}: Deze gebruiker is verantwoordelijk voor het aanmaken van de docent accounts en het toekennen van vakken aan docenten. 
\item[-] \textit{Programma beheerder}: Deze gebruiker is verantwoordelijk voor het aanmaken van vakken, programma�s en faculteiten.

\item[-] \textit{Faciliteit beheerder}: Deze gebruiker is verantwoordelijk voor het aanmaken van de faciliteiten. Deze faciliteiten zullen vervolgens door roosterplanner worden verdeeld over de verschillende vakken in overeenkomst met de beperkingen opgelegd door de docent van het vak.

\item[-] \textit{Docent}: Gebruiker waarvan een lijst van gegeven vakken wordt bijgehouden. De docent is verantwoordelijk voor het beheer van de vakken die hem door een programma beheerder zijn toebedeeld. De docent is ook instaat beperkingen op te leggen de in rekening moeten worden gebracht tijdens het opstellen van de lessenroosters.
\item[-] \textit{Student}: Gebruiker waarvan een lijst van gevolgde programma�s en vakken wordt bijgehouden. Deze lijst word aan de student toegekend door een student beheerder maar kan door de student zelf worden aangevuld. Deze aanvullingen worden echter niet in rekening gebracht tijdens het opstellen van de lessenroosters. Een student kan, naast de gegevens waarover een gast beschikt, ook nog een persoonlijk lessenrooster opvragen dat  wordt opgesteld aan de hand van zijn vakkenlijst.
\item[-] \textit{Gast}: Gebruiker waarvan geen gebruikergegevens worden bijgehouden in de database op de server. Deze gebruiker heeft enkel toegang tot de geplande lessenroosters. Deze kan hij opvragen op basis van vak, student, programma, docent en semester.
\item[-] \textit{Onbekende gebruiker}: Gebruiker die zich nog niet heeft ge�dentificeerd tegenover het systeem. Deze gebruiker heeft enkel toegang tot de aanmeldpagina van de website waarbij hij kan kiezen zich aan te melden of de rest van de site te betreden als gast.
\end{itemize}

\subsubsection{Identificatie}

\begin{itemize}
\item[A.1] Aanmelden \\
Categorie: \textit{noodzakelijke invoerfunctionaliteit} \\
Gebruikertype: \textit{onbekende gebruiker} \\
Omschrijving: Doorsturen van gebruikersnaam en wachtwoord om toegang te krijgen tot de account gebonden gegevens en functionaliteiten \\[-3mm]

\item[A.3] Aanmelden als gast \\
Categorie: \textsl{noodzakelijke invoerfunctionaliteit} \\
Gebruikertype: \textsl{onbekende gebruiker} \\
Omschrijving: De website betreden zonder een account. Zie gast voor meer details. \\[-3mm]

\item[A.3] Afmelden \\
\textit{Categorie: }noodzakelijke invoerfunctionaliteit \\
\textit{Gebruikertype:} iedereen behalve onbekende gebruiker \\
Omschrijving: De site verlaten \\[-3mm]
\end{itemize}

\subsubsection{Opvragen van gegevens}
\begin{itemize}
\item[B.1] Opvragen van faculteit \\
Categorie: \textit{noodzakelijke uitvoerfunctionaliteit} \\
Gebruikertype: \textit{iedereen behalve onbekende gebruiker} \\
Omschrijving: Laat toe na te gaan welke programma�s zijn gebonden aan welke faculteit \\[-3mm]

\item[B.2] Opvragen van programma \\
Categorie: \textit{noodzakelijke uitvoerfunctionaliteit} \\
Gebruikertype: \textit{iedereen behalve onbekende gebruiker} \\
Omschrijving: Laat toe na te gaan welke vakken behoren tot welk programma \\[-3mm]

\item[B.3] Opvragen van vak \\
Categorie: \textit{noodzakelijke uitvoerfunctionaliteit} \\
Gebruikertype: \textit{iedereen behalve onbekende gebruiker} \\
Omschrijving: Geeft toegang tot vak gebonden gegevens \\[-3mm]

\item[B.4] Opvragen van student \\
Categorie: \textit{noodzakelijke uitvoerfunctionaliteit} \\
Gebruikertype: \textit{iedereen behalve onbekende gebruiker} \\
Omschrijving: Laat toe de gevolgde programma�s en vakken van een student na te gaan \\[-3mm]

\item[B.5] Opvragen van docent  \\
Categorie: \textit{noodzakelijke uitvoerfunctionaliteit} \\
Gebruikertype: \textit{docent} \\
Omschrijving: Laat toe na te gaan voor welke vakken een docent verantwoordelijk is \\[-3mm]

\item[B.6] Opvragen van lessenrooster op programma niveau \\
Categorie: \textit{noodzakelijke uitvoerfunctionaliteit} \\
Gebruikertype: \textit{iedereen behalve onbekende gebruiker} \\
Omschrijving: Geeft een rooster weer met alle vakken van een programma\\[-3mm]

\item[B.7] Opvragen van lessenrooster op vak niveau \\
Categorie: \textit{mogelijke uitvoerfunctionaliteit} \\
Gebruikertype: \textit{iedereen behalve onbekende gebruiker} \\
Omschrijving: Geeft weer wanneer een vak is gepland \\[-3mm]

\item[B.8] Opvragen van lessenrooster op student niveau \\
Categorie: \textit{noodzakelijke uitvoerfunctionaliteit} \\
Gebruikertype: \textit{student} \\
Omschrijving: Geeft het persoonlijk rooster van een student weer \\[-3mm]

\item[B.9] Opvragen van lessenrooster op docent niveau \\
Categorie: \textit{noodzakelijke uitvoerfunctionaliteit} \\
Gebruikertype: \textit{docent} \\
Omschrijving: Geeft het persoonlijk rooster van een docent weer \\[-3mm]
\end{itemize}

\subsubsection{Beheren van vakken}

\begin{itemize}
\item[C.1] Vakken aanmaken \\
Categorie: \textit{noodzakelijke invoerfunctionaliteit} \\
Gebruikertype: \textit{Programma beheerder }\\
Omschrijving: \\[-3mm]

\item[C.2] Vakken verwijderen \\
Categorie: \textit{noodzakelijke invoerfunctionaliteit} \\
Gebruikertype: \textit{Programma beheerder} \\
Omschrijving: \\[-3mm]

\item[C.3] Vakken wijzigen als docent \\
Categorie: \textit{mogelijke invoerfunctionaliteit} \\
Gebruikertype: \textit{Docent} \\
Omschrijving: Laat toe de vakomschrijving te wijzigen \\[-3mm]

\item[C.4] Vakken wijzigen als beheerder \\
Categorie: \textit{noodzakelijke invoerfunctionaliteit} \\
Gebruikertype: \textit{Programma beheerder }\\
Omschrijving: Laat toe de naam, docent en omschrijving van een vak te wijzigen  \\[-3mm]

\item[C.5] Vakken koppelen aan een student \\
Categorie: \textit{noodzakelijke invoerfunctionaliteit} \\
Gebruikertype: \textit{Student beheerder} \\
Omschrijving: Zie student beheerder \\[-3mm]

\item[C.6] Vakken koppelen aan een docent \\
Categorie: \textit{noodzakelijke invoerfunctionaliteit} \\
Gebruikertype: \textit{Docent beheerder} \\
Omschrijving: Zie docent en docent beheerder \\[-3mm]

\item[C.7] Vakken onderverdelen in programma�s \\
Categorie: \textit{noodzakelijke invoerfunctionaliteit} \\
Gebruikertype: \textit{Programma beheerder} \\
Omschrijving: \\[-3mm]

\item[C.8] Vakken inladen uit bestand \\
Categorie: \textit{extra invoerfunctionaliteit} \\
Gebruikertype: \textit{Programma beheerder} \\
Omschrijving: Via een nader te bepalen bestand type vakken inladen in de database om gegevens uit ander database makkelijk te kunnen importeren. \\[-3mm]
\end{itemize}

\subsubsection{Beheren van programma }

\begin{itemize}
\item[D.1] Programma�s aanmaken \\
Categorie: \textit{noodzakelijke invoerfunctionaliteit} \\
Gebruikertype: \textit{Programma beheerder} \\
Omschrijving: Programma�s worden gebruikt om vakken te bundelen \\[-3mm]

\item[D.2] Programma�s inladen uit bestand \\
Categorie: \textit{extra  invoerfunctionaliteit} \\
Gebruikertype: \textit{Programma beheerder} \\
Omschrijving: Via en nader te bepalen bestand type  programma�s inladen in de database om gegevens uit ander database makkelijk te kunnen importeren. \\[-3mm]

\item[D.3] Programma�s verwijderen \\
Categorie: \textit{noodzakelijke invoerfunctionalitei}t \\
Gebruikertype: \textit{Programma beheerder }\\
Omschrijving: \\[-3mm]

\item[D.4] Programma�s wijzigen \\
Categorie: \textit{mogelijke invoerfunctionaliteit} \\
Gebruikertype: \textit{Programma beheerder} \\
Omschrijving: \\[-3mm]

\item[D.5] Programma�s koppelen aan een student \\
Categorie: \textit{noodzakelijke/mogelijk  invoerfunctionaliteit} \\
Gebruikertype: \textit{Programma beheerder, student} \\
Omschrijving: Zie Programma beheerder \\[-3mm]

\item[D.6] Programma�s onderverdelen per faculteit \\
Categorie: \textit{mogelijke invoerfunctionaliteit} \\
Gebruikertype: \textit{Programma beheerder} \\
Omschrijving: \\[-3mm]
\end{itemize}

\subsubsection{Beheren van faculteiten }

\begin{itemize}
\item[E.1] Faculteiten aanmaken \\
Categorie: \textit{mogelijke invoerfunctionaliteit }\\
Gebruikertype: \textit{Programma beheerder} \\
Omschrijving: Faculteiten worden gebruikt om programma�s in te bundelen \\[-3mm]

\item[E.2] Faculteiten inladen uit bestand  \\
Categorie: \textit{mogelijke invoerfunctionaliteit} \\
Gebruikertype: \textit{Programma beheerder} \\
Omschrijving: Via een nader te bepalen bestand type faculteiten inladen in de database om gegevens uit ander database makkelijk te kunnen importeren. \\[-3mm]

\item[E.3] Faculteiten verwijderen \\
Categorie: \textit{mogelijke invoerfunctionaliteit} \\
Gebruikertype: \textit{Programma beheerder }\\
Omschrijving: Gebruiker kan kiezen of programma�s en vakken van een faculteit mee worden verwijderd \\[-3mm]

\item[E.4] Faculteiten wijzigen \\
Categorie: \textit{mogelijke invoerfunctionaliteit} \\
Gebruikertype: \textit{Programma beheerder }\\
Omschrijving: Bepalen welke programma�s tot welke faculteit behoren \\[-3mm]
\end{itemize}

\subsubsection{Beheren van  faciliteiten}

\begin{itemize}
\item[F.1] Faciliteiten aanmaken \\
Categorie: \textit{mogelijke invoerfunctionaliteit} \\
Gebruikertype: \textit{Faciliteiten beheerder} \\
Omschrijving: Faciliteiten omvatten zowel locaties (vb lokaal) als benodigdheden (vb projector) \\[-3mm]

\item[F.2] Faciliteiten inladen uit bestand \\
Categorie: \textit{extra  invoerfunctionaliteit} \\
Gebruikertype: \textit{Faciliteiten beheerder} \\
Omschrijving: Via een nader te bepalen bestand type faciliteiten inladen in de database om gegevens uit ander database makkelijk te kunnen importeren. \\[-3mm]

\item[F.3] Faciliteiten verwijderen \\
Categorie: \textit{mogelijke invoerfunctionaliteit} \\
Gebruikertype: \textit{Faciliteiten beheerder}  \\
Omschrijving: Als een faciliteit wordt verwijderd moeten alle afhankelijke beperkingen ook verwijderd \\[-3mm]

\item[F.4] Faciliteiten wijzigen  \\
Categorie: \textit{mogelijke invoerfunctionaliteit} \\
Gebruikertype: \textit{Faciliteiten  beheerder} \\
Omschrijving: \\[-3mm]
\end{itemize}

\subsubsection{Beheren van accounts}

\begin{itemize}
\item[G.1] Gebruikers aanmaken \\
Categorie: \textit{noodzakelijke invoerfunctionaliteit} \\
Gebruikertype: \textit{Account beheerder} \\
Omschrijving: \\[-3mm]

\item[G.2] Gebruikers inladen uit bestand \\
Categorie: \textit{extra invoerfunctionaliteit} \\
Gebruikertype: \textit{Account beheerder} \\
Omschrijving: Via een nader te bepalen bestand type gebruikers inladen in de database om gegevens uit ander database makkelijk te kunnen importeren. \\[-3mm]

\item[G.3] Gebruikers verwijderen \\
Categorie: \textit{noodzakelijke invoerfunctionaliteit} \\
Gebruikertype: \textit{Account beheerder} \\
Omschrijving: Alle gegevens van de gebruiker worden uit de database verwijderd \\[-3mm]

\item[G.4] Gebruikers wijzigen \\
Categorie: \textit{mogelijke invoerfunctionaliteit} \\
Gebruikertype: \textit{Account beheerder} \\
Omschrijving: \\[-3mm]

\item[G.5] Gebruikers blokkeren \\
Categorie: \textit{mogelijke  invoerfunctionaliteit} \\
Gebruikertype: \textit{Account beheerder}\\
Omschrijving: Een geblokkeerde gebruiker krijgt geen toegang tot het systeem tijdens het aanmelden \\[-3mm]

\item[G.6] Gebruikertypes aanmaken \\
Categorie: \textit{mogelijke invoerfunctionaliteit} \\
Gebruikertype: \textit{Account beheerder} \\
Omschrijving: Laat toe functionaliteit te bundelen in gebruikertypes op maat.\\[-3mm]

\item[G.7] Gebruikertypes verwijderen \\
Categorie: \textit{mogelijke invoerfunctionaliteit} \\
Gebruikertype: \textit{Account beheerder} \\
Omschrijving: Sommige gebruikertypes zijn beschermd tegen verwijdering om verdere werking van de software te garanderen. \\[-3mm]

\item[G.8] Gebruikertypes wijzigen \\
Categorie: \textit{mogelijke invoerfunctionaliteit} \\
Gebruikertype: \textit{Account beheerder} \\
Omschrijving: Sommige gebruikertypes zijn beschermd tegen aanpassingen om verdere werking van de software te garanderen. \\[-3mm]
\end{itemize}

\subsubsection{Beheren van beperkingen}

\begin{itemize} 
\item[H.1] Tijdsbeperking aanmaken  \\
Categorie: \textit{noodzakelijke  invoerfunctionaliteit} \\
Gebruikertype: \textit{Rooster beheerder, docent} \\
Omschrijving: een tijdsbeperking legt op waneer een vak wel of niet kan worden geplant \\[-3mm]

\item[H.2] Tijdsbeperking verwijderen \\
Categorie: \textit{noodzakelijke invoerfunctionaliteit} \\
Gebruikertype: \textit{Rooster beheerder, docent} \\
Omschrijving: \\[-3mm]

\item[H.3] Tijdsbeperking wijzigen \\
Categorie: \textit{noodzakelijke invoerfunctionaliteit} \\
Gebruikertype: \textit{Rooster beheerder, docent} \\
Omschrijving: \\[-3mm]

\item[H.4] Faciliteitbeperking aanmaken \\
Categorie: \textit{mogelijke invoerfunctionaliteit} \\
Gebruikertype: \textit{Rooster beheerder, docent} \\
Omschrijving: een faciliteitbeperking legt op welke faciliteiten er nodig zijn voor een vak \\[-3mm]

\item[H.5] Faciliteitbeperking verwijderen \\
Categorie: \textit{mogelijke invoerfunctionaliteit} \\
Gebruikertype: \textit{Rooster beheerder, docent} \\
Omschrijving: \\[-3mm]

\item[H.6] Faciliteitbeperking wijzigen \\
Categorie: \textit{mogelijke invoerfunctionaliteit} \\
Gebruikertype: \textit{Rooster beheerder, docent} \\
Omschrijving: \\[-3mm]
\end{itemize}

\subsubsection{Roosterplanner}

\begin{itemize}
\item[I.1] Configureren van roosterplanner \\
Categorie: \textit{mogelijke invoerfunctionaliteit} \\
Gebruikertype: \textit{Software beheerder, Rooster beheerder} \\
Omschrijving: Bepaald met welke gegevens de roosterplanner moet rekening houden \\[-3mm]

\item[I.2] Starten van roosterplanner \\
Categorie: \textit{mogelijke invoerfunctionaliteit} \\
Gebruikertype: \textit{Rooster beheerder} \\
Omschrijving: Data noodzakelijk voor de planning mag na het starten niet meer worden aangepast .Het starten van de roosterplanner bevriest dus sommige gegevens tot zijn planningstaak is gestopt. \\[-3mm]

\item[I.3] Stoppen van roosterplanner \\
Categorie: \textit{mogelijke invoerfunctionaliteit} \\
Gebruikertype: \textit{Rooster beheerder} \\
Omschrijving: Het handmatig stoppen van de roosterplanner. De planner kan alter weer worden gestart om zijn taak verder te zetten. \\[-3mm]

\item[I.4] Status van de roosterplanner opvragen \\
Categorie: \textit{mogelijke uitvoerfunctionaliteit} \\
Gebruikertype: \textit{Rooster beheerder} \\
Omschrijving: Het handmatig stoppen van de roosterplanner. De planner kan alter weer worden gestart om zijn taak verder te zetten. \\[-3mm]

\item[I.5] Melden van conflicten in het rooster \\
Categorie: \textit{mogelijke verwerkingsfunctionaliteit} \\
Gebruikertype: \textit{Rooster beheerder} \\
Omschrijving: Als de rooster planner geen rooster kan vinden dat voldoet meldt hij dit als een conflict \\[-3mm]
\end{itemize}

\subsubsection{Beheren van lessenroosters}
\begin{itemize}
\item[J.1] Lessenroosters bereken \\
Categorie: \textit{mogelijke verwerkingsfunctionaliteit} \\
Gebruikertype: \textit{Roosterbeheerder} \\
Omschrijving: De rooster planner berekend automatisch roosters die voldoen aan de opgegeven beperkingen. \\[-3mm]

\item[J.2] Lessenroosters aanmaken \\
Categorie: \textit{mogelijke invoerfunctionaliteit} \\
Gebruikertype: \textit{Rooster beheerder} \\
Omschrijving: \\[-3mm]

\item[J.3] Lessenroosters verwijderen \\
Categorie: \textit{mogelijke invoerfunctionaliteit} \\
Gebruikertype: \textit{Rooster beheerder }\\
Omschrijving: \\[-3mm]

\item[J.4] Lessenroosters wijzigen \\
Categorie: \textit{mogelijke invoerfunctionaliteit} \\
Gebruikertype: \textit{Rooster beheerder} \\
Omschrijving: Lessen roosters kunnen handmatig worden aangespast \\[-3mm]
\end{itemize}

\subsubsection{Beheren van logboek }

\begin{itemize}
\item[K.1] Logboek aanmaken \\
Categorie: \textit{extra  invoerfunctionaliteit} \\
Gebruikertype: \textit{Software beheerder} \\
Omschrijving: Als een logboek is aangemaakt zullen belangrijke gebeurtenissen in het systeem worden bijgehouden \\[-3mm]

\item[K.2] Logboek bekijken \\
Categorie: \textit{extra uitvoerfunctionaliteit} \\
Gebruikertype: \textit{Software beheerder} \\
Omschrijving: \\[-3mm]

\item[K.3] Gebeurtenissen registreren \\
Categorie: \textit{extra  verwerkingsfunctionaliteit} \\
Gebruikertype: \textit{Software beheerder} \\
Omschrijving: \\[-3mm]

\item[K.4] Gebeurtenissen ongedaan maken \\
Categorie: \textit{extra invoerfunctionaliteit} \\
Gebruikertype: \textit{Software beheerder} \\
Omschrijving: \\[-3mm]

\end{itemize}

\subsubsection{Overige}

\begin{itemize}
\item[X.1] Aanmaken van de software databasestructuur via SQL \\
Categorie: \textit{noodzakelijke verwerkingsfunctionaliteit} \\
Gebruikertype: \textit{Software beheerder} \\
Omschrijving: De software brengt zelf een datastructuur aan die nodig is voor het opslaan van de gegevens  \\[-3mm]

\item[X.2] Aanpassen van de �look en feel� van de website \\
Categorie: \textit{extra invoerfunctionaliteit} \\
Gebruikertype: \textit{Software beheerder} \\
Omschrijving: Dit wordt mogelijk gemaakt door et toegankelijk maken van de opmaak bestanden. \\[-3mm]
\end{itemize}

\subsubsection{Opmerking over User Classes}

We verkiezen een methode waarbij de rechten van een account niet volledig vast liggen door zijn user class, maar veel losser kunnen bepaald worden. De user classes uit de vorige puntjes zijn templates, richtwaarden waar van afgeweken kan worden. De structuur moet dus per feature beschreven worden in een latere versie van het SRS.

\\ 
\\

%ZJEF

\subsection{Performance requirements}

Er zijn geen specifieke vereisten in verband met de snelheid van de software. Het is echter wel duidelijk uit de aard van het project, dat er een groot aantal gebruikers tegelijk de webinterface (en met gevolg de databases) moet kunnen consulteren.

\subsection{Design constraints}

De opdrachtgever heet de omgeving waarin de software moet draaien gespecificeerd.\\
De opgelegde beperkingen, met betrekking op het design, zijn de volgende:

\begin{itemize}
\item[.] Systeem moet draaien op een linux server, meer bepaald Wilma (http://wilma.vub.ac.be/) \\[-5mm]
\item[.] Uitsluitend gebruik van open source software \\[-5mm]
\item[.] User interactie gebeurt via een gebruiksvriendelijke grafische web interface met bovengenoemde server \\[-5mm]
\item[.] Flexibiliteit van verschillende parameters instelbaar door de gebruiker \\[-5mm]
\end{itemize}

%\subsection{Software system attributes}

% \subsubsection{Reliability}
% This should specify the factors required to establish the required reliability of the software system at time of delivery.


% \subsubsection{Availability}
% This should specify the factors required to guarantee a defined availability level for the entire system such as checkpoint, recovery, and restart.


\subsubsection{Security}
%This should specify the factors that protect the software from accidental or malicious access, use, modification, destruction, or disclosure. Specific requirements in this area could include the need to
% a) Utilize certain cryptographical techniques;
% b) Keep specific log or history data sets;
% c) Assign certain functions to different modules;
% d) Restrict communications between some areas of the program;
% e) Check data integrity for critical variables.

%Als de gebruiker zich aanmeldt, wordt een code gegenereerd voor die gebruiker (random nummer) die lokaal bijgehouden wordt door de gebruiker.\\
%Bij elk commando van de gebruiker wordt de code meegegeven.\\
%Op de server wordt een tijdelijke lijst bijgehouden die de code aan accounts verbindt. (en inlogtijd,... om ze na een tijd weg te kunnen smijten als de gebruiker niet letterlijk uitlogt). \\
%Bij elk commando van de gebruiker wordt de code in de lijst opgezocht, de rechten van de bijhorende gebruiker gecontroleerd en al dan niet kan het commando uitgevoerd worden.\\[5mm]
%Logboek van aanpassingen door Managers wordt bijgehouden.\\

\textbf{User access restriction}

Verschillende gebruikers krijgen verschillende rechten toegewezen. Deze bepalen tot welke informatie en tools hij toegang krijgt. Deze rechten zijn gekoppeld aan de account van deze gebruiker. Na het inloggen zullen enkel de informatie en tools waarvoor de gebruiker gemachtigd is, getoond worden. Om verdere beveiliging te verzekeren, wordt ook de communicatie met de server beveiligd. Bij het inloggen krijgt de webinterface van de gebruiker een access code toegestuurd, op dat moment gegenereerd door de server. Deze houdt bij welke rechten bij deze code horen. De webinterface stuurt de verkregen code mee door met elke instructie naar de server, waarop deze kan controleren of de gebruiker gemachtigd is om de desbetreffende instructie uit te voeren.\\

\textbf{Data integriteit}

Enerzijds zal er de mogelijkheid geleverd worden, aan de daarvoor gemachtigde gebruiker(s), om op de server een back-up van de databases (zoals accounts, leslokalen, vakken,...) te maken en desnoods een rollback uit te voeren.
Anderzijds zal er op de server een logboek bijgehouden worden met de aanpassingen aan de databases, die door verschillende gebruikers gemaakt kunnen worden.\\

\textbf{Account security}

Account paswoorden zullen, met een nog nader te bepalen, ge\"{e}ncrypteerd verstuurd en opgeslagen
worden.

% \subsubsection{Maintainability}
% This should specify attributes of software that relate to the ease of maintenance of the software itself. There may be some requirement for certain modularity, interfaces, complexity, etc. Requirements should not be placed here just because they are thought to be good design practices.


% \subsubsection{Portability}
% This should specify attributes of software that relate to the ease of porting the software to other host machines and/or operating systems. This may include the following:
% a) Percentage of components with host-dependent code;
% b) Percentage of code that is host dependent;
% c) Use of a proven portable language;
% d) Use of a particular compiler or language subset;
% e) Use of a particular operating system.

% \subsection{Other requirements}

%\newpage
%\section*{Index}




%\begin{figure}
%\begin{center}
%\includegraphics[width=\textwidth]{Bangladesh}
%\caption*{Bangladesh}
%\end{center}
%\end{figure}

%\begin{tabular}[t]{llll}
%Mandi & Katholiek & 25\% & Platte neus -- spleetogen \\
%Gohli & Moslims   & 75\% & Grote ogen -- scherpere neus \\
%Kooch & Hindu     & 1\%  & Mengeling van de twee \\
%\end{tabular}
%\\[5mm]

%\begin{itemize}
%\item[.] Slaapmatje \\[-3mm]
%\item[.] Kleren: 3 korte broeken, 6 onderbroeken, 4 T-shirts\\[-3mm]
%\item[.] Longi: voor 's avonds en als pyjama\\[-3mm]
%\item[.] Tandenborstel en twee tubes tandpasta\\[-3mm]
%\end{itemize}


 \end{document}
