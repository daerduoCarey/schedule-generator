\documentclass{article}
\usepackage[dutch]{babel}
\usepackage[official]{eurosym}
%\usepackage{graphicx}
%\usepackage{epic,eepic}
%\usepackage{mathtools}
%\usepackage{caption}
\usepackage{hyperref}
\usepackage{textcomp}
\title{Software Requirements Specifications voor Schedule-Generator}
\author{Matthias Caenepeel \and Adam Cooman \and Alexander De Cock \and Zjef Van de Poel}
\date{03 april 2011 Versie 2.0} 

\makeatletter
\renewcommand\paragraph{
   \@startsection{paragraph}{4}{0mm}
      {-\baselineskip}
      {.5\baselineskip}
      {\normalfont\normalsize\bfseries}}
\makeatother

\makeatletter
\renewcommand\subparagraph{
   \@startsection{subparagraph}{5}{0mm}
      {-\baselineskip}
      {.5\baselineskip}
      {\normalfont\normalsize\bfseries}}
\makeatother

\begin{document}

\maketitle

\newpage

% \newpage
% Signature page

% \newpage

\section*{Aanpassingsgeschiedenis}
\begin{itemize}
\item[.] 23/2/2011 versie 0.1: Aanmaak document, toevoeging  
\item[.] 25/2/2011 versie 0.2: Toevoeging tekst vanaf Performance Requirements tot einde 
\item[.] 28/2/2011 versie 1.0: Toevoeging functionele vereisten, verbeteringen doorgevoerd 
\item[.] 17/3/2011 versie 1.1: Opmerkingen van opdrachtgever in acht genomen en aanpassingen doorgevoerd waar nodig
\item[.] 03/4/2011 versie 2.0: Volledige revisie van het document
\end{itemize}

\section*{Nog aan te passen}
\begin{itemize}
	\item[.] De inhoud van secties 3.1.1 en 3.1.2 horen eigenlijk thuis in een SDD. Alles in heel sectie 3 moet geformuleerd worden in termen van requirements, dus met een vaste structuur, een unieke ID, pre- en postcondities etc. Een requirement moet zo geformuleerd worden dat ze als het ware kan afgevinkt worden op het einde. \\
	\item[.] 3.2.12 "Roosterplanner" lijkt ons een heel belangrijk en non-triviaal onderdeel van de opgave. Het lijkt ons hier aangewezen iets gedetailleerdere (en eventueel opgesplitste) requirements te voorzien. Bvb I.5 "Melden van conflicten in het rooster" zou toch specifieker kunnen. Tot hoeveel detail wil je minimaal voorzien, gaat hij exact aan kunnen geven waar en waarom een conflict zich voordoet (soms kan dit complex zijn). \\
\end{itemize}

%\section*{Preface}

\newpage
%\tableofcontents

\newpage

\section{Inleiding}
\subsection{Doelstelling}
% Delineate the purpose of the SRS.
% Specify the intended audience for the SRS.

Het doel van de SRS is om een overzicht te geven van alle functionaliteiten die er moeten voorzien worden. \\
Het doelpubliek is het team dat aan het project werkt en de docent die het team begeleidt en evalueert. \\

\subsection{Onderwerpen}
% Identify the software product(s) to be produced by name (e.g., Host DBMS, Report Generator, etc.);
% Explain what the software product(s) will, and, if necessary, will not do;
% Describe the application of the software being specified, including relevant benefits, objectives, and goals;
% Be consistent with similar statements in higher-level specifications (e.g., the system requirements speci�cation), if they exist.

\begin{description}
\item[Schedule generator] deze software (geschreven in Java) zal in staat zijn om een lessenrooster samen te stellen dat aan bepaalde voorwaarden voldoet.  \\
\item[Website] dit is de software die de site omvat waarop de gebruikers zullen werken. \\
\item[Database] deze zal de informatie over het lessenrooster bijhouden en ordenen. \\
\item[Servelets] deze software zal ervoor zorgen dat de site doet wat de gebruiker vraagt. 
\end{description}

%\subsection{Definitions, acronyms and abbreviations}
%\subsection{references}
%Provide a complete list of all documents referenced elsewhere in the SRS;
%Identify each document by title, report number (if applicable), date, and publishing organization;
%Specify the sources from which the references can be obtained.

\subsection{Overzicht}
% Describe what the rest of the SRS contains;
% Explain how the SRS is organized.

De rest van de SRS zal de software vereisten van de schedule generator verder uitdiepen. De verschillende gebruikers zullen gedefinieerd worden en de functionaliteiten die die gebruikers krijgen zullen opgesomd en besproken worden.

\newpage

\section{Algmene beschrijving}
% \subsection{Product perspective}

% blokdiagram van programma en verschillende softwareinterfaces die we gebruiken
% MySQL, Tomcat, Java Runtime Enviroment voor generator zelf ...

% \subsubsection{System interfaces}
% \subsubsection{User interfaces}
% \subsubsection{Hardware interfaces}
% \subsubsection{Software interfaces}
% \subsubsection{Communications interfaces}
% \subsubsection{Memory}
% \subsubsection{Operations}
% \subsubsection{Site adaptation requirements}

\subsection{Functionaliteiten}
%samenvatting van de belangrijkste functionaliteiten. \\

Gebruikers krijgen een gepersonaliseerde account waarop zij hun eigen lessenrooster alsook dat van andere richtingen kunnen opvragen (en in het geval van sommige gebruikers kunnen aanpassen).\\
Het programma bevat een database waarin informatie over de verschillende vakken wordt bijgehouden (wie is de docent, hoeveel studenten zijn er ingeschreven, waar wordt het gedoceerd,...)\\
Het bevat ook een schedule generator die in staat is een lessenrooster te genereren die aan bepaalde voorwaarden voldoet.

\subsection{Gebruikers omschrijving}

De gebruikers zijn studenten die in staat om zijn om de verschillende lessenroosters te bekijken en docenten die in staat zijn het rooster op te vragen, maar evenwel aanpassingen te maken aan de informatie van de vakken waarvoor zij verantwoordelijk zijn.
Daarnaast zijn er ook verschillende klassen beheerders die verantwoordelijk zijn voor groepen vakken, groepen studenten, gebouwen en het lessenrooster zelf. Deze beheerders moeten de nodige aanpassingen ook op de website kunnen doorvoeren.

\subsection{Ontwerpbeperkingen}
De opdracht gever heeft de beperkingen voor de ontwikkeling van het project in grote lijnen vastgelegd aan de hand van een opdrachtbeschrijving die terug kan worden gevonden in het SPMP.
Hieronder volgt de lijst van de beperkingen die in rekening worden gebracht:
\begin{itemize}
	\item[-] Alle geproduceerde en gebruikte code moet open source zijn, evenals de software tools die worden aangewend voor dit project
	\item[-] Het eindproduct moet draaien op een linux server, meer bepaald Wilma (http://wilma.vub.ac.be/)
	\item[-] Het project moet ontwikkeld worden in een object geori�nteerde programmeer taal. Het team heeft er voor gekozen JAVA te gebruiken omdat het daarmee het meeste ervaring heeft.
	\item[-] De gebruikerinteractie gebeurt via een gebruiksvriendelijke grafische web interface.
	\item[-] De configuratie van het eindproduct moet op een eenvoudige wijze kunnen worden uitgevoerd door de gebruiker.
\end{itemize}

% \subsection{Assumptions and dependencies}

\newpage

\section{Specifieke vereisten}

\subsection{Functionele Vereisten}

In deze subsectie van het SRS  wordt een overzicht gegeven van alle functionaliteiten van de software. Om de ontwikkeling van de functionaliteiten tijdens het productie te kunnen opvolgen, wordt elke functionaliteit voorzien van een naam, code en omschrijving. Daarnaast worden de functionaliteiten ook nog eens opgedeeld in verschillende categorie�n en indien mogelijk toegekend aan een bepaald gebruikertype.

\subsubsection{Categorie op basis van prioriteit}
Enerzijds kan men de functionaliteiten opdelen volgens hun prioriteit. Dit leidt tot onderstaande categorie�n:
\begin{itemize}	
\item[-] \textit{noodzakelijke functionaliteiten}: functionaliteiten waarvan de aanwezigheid in het eindproduct wordt gegarandeerd omdat ze noodzakelijk zijn voor een minimale werking van het systeem. Deze functionaliteiten hebben de hoogste prioriteit tijdens de verwezenlijking van het project. \\
\item[-] \textit{mogelijke functionaliteiten}: functionaliteiten waarvan de uitvoering haalbaar is maar zonder garantie op aanwezigheid in het eindproduct. Deze functionaliteiten hebben een matige prioriteit \\
\item[-] \textit{extra functionaliteiten}: functionaliteiten waarvoor de prioriteit laag is. \\
\end{itemize}

\subsubsection{Categorie op basis van datapad}
Anderzijds kan men functionaliteiten ook opdelen volgens het pad die de data tijdens hun uitvoering volgt. Dit geeft aanleiding tot volgende categorie�n

\begin{itemize}
\item[-] \textit{Invoerfunctionaliteiten}:  alle functionaliteiten die de gebruiker toelaten informatie op te sturen naar de database op de server. 
\item[-] \textit{Uitvoerfunctionaliteiten}: alle functionaliteiten die de gebruiker toelaten informatie op te vragen uit de database op de server.
\item[-] \textit{Verwerkingsfunctionaliteiten}: alle functionaliteiten die gegevens, zonder tussenkomst van de gebruiker, uit de database verwerken en de daarbij bekomen resultaten terug schrijven naar de database op de server.
\end{itemize}

\subsubsection{Gebruikertype}
Om bepaalde functionaliteiten te kunnen plaatsen in een context wordt het gebruikertype dat toegang heeft tot deze functionaliteit indien mogelijk vermeld. Vandaar volgt hier een korte beschrijving van de gebruikertypes:
\begin{itemize}
\item[-] \textit{Software beheerder}: Verantwoordelijke voor het beheren technische aspecten van de software. Deze omvatten o.a. instaleren van de software, laten aanmaken van de databasestructuur en het configureren van de lessenroosterplanner. Daarnaast heeft deze gebruiker ook toegang tot een logboek en kan hij bepaalde gebeurtenissen ongedaan te maken.
\item[-] \textit{Account beheerder}: Deze gebruiker kan de omschrijving van bestaande gebruikertypes wijzigen en nieuwe gebruikertypes defini�ren. Hij is ook verantwoordelijk voor het aanmaken van de andere beheerder accounts.
\item[-] \textit{Rooster beheerder}: Deze gebruiker is instaat lessenrooster manueel aan te passen, lessenroosters door de roosterplanner te laten genereren en algemene beperkingen voor de roosterplanner in te stellen (bvb: feestdagen). De rooster beheerder zal ook eventuele conflict situaties in het rooster moeten oplossen. 
\item[-] \textit{Student beheerder}: Deze gebruiker is verantwoordelijk voor het aanmaken van de studenten accounts en het toekennen van programma�s en vakken aan studenten. De vakken toegewezen door de student beheerder zullen in rekening worden gebracht tijdens het opstellen van de lessenroosters. 
\item[-] \textit{Docent beheerder}: Deze gebruiker is verantwoordelijk voor het aanmaken van de docent accounts en het toekennen van vakken aan docenten. 
\item[-] \textit{Programma beheerder}: Deze gebruiker is verantwoordelijk voor het aanmaken van vakken, programma�s en faculteiten.

\item[-] \textit{Faciliteit beheerder}: Deze gebruiker is verantwoordelijk voor het aanmaken van de faciliteiten. Deze faciliteiten zullen vervolgens door roosterplanner worden verdeeld over de verschillende vakken in overeenkomst met de beperkingen opgelegd door de docent van het vak.

\item[-] \textit{Docent}: Gebruiker waarvan een lijst van gegeven vakken wordt bijgehouden. De docent is verantwoordelijk voor het beheer van de vakken die hem door een programma beheerder zijn toebedeeld. De docent is ook instaat beperkingen op te leggen de in rekening moeten worden gebracht tijdens het opstellen van de lessenroosters.
\item[-] \textit{Student}: Gebruiker waarvan een lijst van gevolgde programma�s en vakken wordt bijgehouden. Deze lijst word aan de student toegekend door een student beheerder maar kan door de student zelf worden aangevuld. Deze aanvullingen worden echter niet in rekening gebracht tijdens het opstellen van de lessenroosters. Een student kan, naast de gegevens waarover een gast beschikt, ook nog een persoonlijk lessenrooster opvragen dat  wordt opgesteld aan de hand van zijn vakkenlijst.
\item[-] \textit{Gast}: Gebruiker waarvan geen gebruikergegevens worden bijgehouden in de database op de server. Deze gebruiker heeft enkel toegang tot de geplande lessenroosters. Deze kan hij opvragen op basis van vak, student, programma, docent en semester.
\item[-] \textit{Onbekende gebruiker}: Gebruiker die zich nog niet heeft ge�dentificeerd tegenover het systeem. Deze gebruiker heeft enkel toegang tot de aanmeldpagina van de website waarbij hij kan kiezen zich aan te melden of de rest van de site te betreden als gast.
\end{itemize}

\subsubsection{Opmerking over de gebruikertypes}
De reeds besproken gebruikertypes zullen standaard aanwezig zijn bij het afgeleverde product. Het echter mogelijk zijn voor de eindgebruiker om nieuwe gebruikertypes te defini�ren of de rechte van bestaande gebruikertypes aan te passen. Onderstaande verdeling van functionaliteiten over de gebruikertypes is dus enkel van toepassing op de standaard instellingen van het product. Voor meer informatie over de uitvoering van de gebruikertypes wordt verwezen naar het SDD.

\subsubsection{Identificatie}

\begin{itemize}
\item[A.1] Aanmelden \\
Status: \textit{Voltooid} \\
Categorie: \textit{noodzakelijke invoerfunctionaliteit} \\
Gebruikertype: \textit{onbekende gebruiker} \\
Omschrijving: Doorsturen van gebruikersnaam en wachtwoord om toegang te krijgen tot de account gebonden gegevens en functionaliteiten \\[-3mm]

\item[A.2] Aanmelden als gast \\
Status: \textit{Voltooid} \\
Categorie: \textsl{noodzakelijke invoerfunctionaliteit} \\
Gebruikertype: \textsl{onbekende gebruiker} \\
Omschrijving: De website betreden zonder een account. Zie gast voor meer details. 

\item[A.3] Afmelden \\
Status: \textit{Voltooid} \\
Categorie: \textit{noodzakelijke invoerfunctionaliteit} \\
Gebruikertype: \textit{iedereen behalve onbekende gebruiker} \\
Omschrijving: De site verlaten en de huidige sessie verwijderen
\end{itemize}

\subsubsection{Opvragen van gegevens}
\begin{itemize}
\item[B.1] Opvragen van faculteit \\
Status: \textit{Voltooid} \\
Categorie: \textit{noodzakelijke uitvoerfunctionaliteit} \\
Gebruikertype: \textit{iedereen behalve onbekende gebruiker} \\
Omschrijving: Laat toe na te gaan welke programmas zijn gebonden aan welke faculteit 

\item[B.2] Opvragen van programma \\
Status: \textit{Voltooid} \\
Categorie: \textit{noodzakelijke uitvoerfunctionaliteit} \\
Gebruikertype: \textit{iedereen behalve onbekende gebruiker} \\
Omschrijving: Laat toe na te gaan welke vakken behoren tot welk programma 

\item[B.3] Opvragen van vak \\
Status: \textit{Voltooid} \\
Categorie: \textit{noodzakelijke uitvoerfunctionaliteit} \\
Gebruikertype: \textit{iedereen behalve onbekende gebruiker} \\
Omschrijving: Geeft toegang tot vak gebonden gegevens \\[-3mm]

\item[B.4] Opvragen van student \\
Status: \textit{Voltooid} \\
Categorie: \textit{noodzakelijke uitvoerfunctionaliteit} \\
Gebruikertype: \textit{iedereen behalve onbekende gebruiker} \\
Omschrijving: Laat toe de gevolgde programma�s en vakken van een student na te gaan 

\item[B.5] Opvragen van docent  \\
Status: \textit{Voltooid} \\
Categorie: \textit{noodzakelijke uitvoerfunctionaliteit} \\
Gebruikertype: \textit{docent} \\
Omschrijving: Laat toe na te gaan voor welke vakken een docent verantwoordelijk is 

\item[B.6] Opvragen van lessenrooster op programma niveau \\
Status: \textit{Niet voltooid} \\
Categorie: \textit{noodzakelijke uitvoerfunctionaliteit} \\
Gebruikertype: \textit{iedereen behalve onbekende gebruiker} \\
Omschrijving: Geeft een rooster weer met alle vakken van een programma. Gegevens zijn aanwezig maar kunnen niet via de site worden opgevraagd.

\item[B.7] Opvragen van lessenrooster op vak niveau \\
Status: \textit{Niet voltooid} \\
Categorie: \textit{mogelijke uitvoerfunctionaliteit} \\
Gebruikertype: \textit{iedereen behalve onbekende gebruiker} \\
Omschrijving: Geeft weer wanneer een vak is gepland. Gegevens zijn aanwezig maar kunnen niet via de site worden opgevraagd.

\item[B.8] Opvragen van lessenrooster op student niveau \\
Status: \textit{Voltooid} \\
Categorie: \textit{noodzakelijke uitvoerfunctionaliteit} \\
Gebruikertype: \textit{student} \\
Omschrijving: Geeft het persoonlijk rooster van een student weer 

\item[B.9] Opvragen van lessenrooster op docent niveau \\
Status: \textit{Voltooid} \\
Categorie: \textit{noodzakelijke uitvoerfunctionaliteit} \\
Gebruikertype: \textit{docent} \\
Omschrijving: Geeft het persoonlijk rooster van een docent weer 

\item[B.10] Opvragen van Account \\
Status: \textit{Voltooid} \\
Categorie: \textit{noodzakelijke uitvoerfunctionaliteit} \\
Gebruikertype: \textit{Account beheerder} \\
Omschrijving: Weergeven van een lijst van alle Accounts.

\item[B.11] Opvragen van de Gebruikertypes \\
Status: \textit{Voltooid} \\
Categorie: \textit{mogelijke uitvoerfunctionaliteit} \\
Gebruikertype: \textit{Account beheerder} \\
Omschrijving:Lijst van alle bestaande gebruikertypes

\item[B.12] Opvragen van Gebouwen en Lokalen \\
Status: \textit{Voltooid} \\
Categorie: \textit{noodzakelijke uitvoerfunctionaliteit} \\
Gebruikertype: \textit{iedereen behalve onbekende gebruiker} \\
Omschrijving: Weergeven van alle gebouwen en hun lokalen.

\item[B.13] Opvragen van Faciliteiten \\
Status: \textit{Voltooid} \\
Categorie: \textit{mogelijke uitvoerfunctionaliteit} \\
Gebruikertype: \textit{Faciliteit beheerder} \\
Omschrijving: Weergeven van alle bestaande faciliteiten 

\item[B.14] Opvragen van Software Configuratie \\
Status: \textit{Niet voltooid} \\
Categorie: \textit{mogelijke uitvoerfunctionaliteit} \\
Gebruikertype: \textit{Software beheerder} \\
Omschrijving: Weergeven van alle configureerbare gegevens zoals paden van bestanden en waarden van software variabelen. Gegevens zijn aanwezig maar kunnen niet via de site worden opgevraagd.

\item[B.15] Opvragen van Beperkingen per type \\
Status: \textit{niet Voltooid} \\
Categorie: \textit{noodzakelijke uitvoerfunctionaliteit} \\
Gebruikertype: \textit{Software Beheerder} \\
Omschrijving: Weergeven van alle ingevoerde Beperkingen. Gegevens zijn aanwezig maar kunnen niet via de site worden opgevraagd.

\item[B.16] Opvragen van Logboek \\
Status: \textit{niet Voltooid} \\
Categorie: \textit{mogelijke uitvoerfunctionaliteit} \\
Gebruikertype: \textit{Software Beheerder} \\
Omschrijving: Weergeven van de logboek gebeurtenissen

\item[B.17] Opvragen van persoonlijke accounts gegevens \\
Status: \textit{Voltooid} \\
Categorie: \textit{noodzakelijke uitvoerfunctionaliteit} \\
Gebruikertype: \textit{iedereen behalve onbekende gebruiker} \\
Omschrijving: weergeven van gegevens gebonden aan een account.

\end{itemize}

\subsubsection{Beheren van vakken}

\begin{itemize}
\item[C.1] Vakken aanmaken \\
Status: \textit{Voltooid} \\
Categorie: \textit{noodzakelijke invoerfunctionaliteit} \\
Gebruikertype: \textit{Programma beheerder }\\
Omschrijving: 

\item[C.2] Vakken verwijderen \\
Status: \textit{Voltooid} \\
Categorie: \textit{noodzakelijke invoerfunctionaliteit} \\
Gebruikertype: \textit{Programma beheerder} \\
Omschrijving: 

\item[C.3] Vakken wijzigen als docent \\
Status: \textit{Voltooid} \\
Categorie: \textit{mogelijke invoerfunctionaliteit} \\
Gebruikertype: \textit{Docent} \\
Omschrijving: Laat toe de vakomschrijving te wijzigen 

\item[C.4] Vakken wijzigen als beheerder \\
Status: \textit{Voltooid} \\
Categorie: \textit{noodzakelijke invoerfunctionaliteit} \\
Gebruikertype: \textit{Programma beheerder }\\
Omschrijving: Laat toe de naam, docent en omschrijving van een vak te wijzigen  

\item[C.5] Vakken koppelen aan een student \\
Status: \textit{Voltooid} \\
Categorie: \textit{noodzakelijke invoerfunctionaliteit} \\
Gebruikertype: \textit{Student beheerder} \\
Omschrijving: Zie student beheerder 

\item[C.6] Vakken koppelen aan een docent \\
Status: \textit{Voltooid} \\
Categorie: \textit{noodzakelijke invoerfunctionaliteit} \\
Gebruikertype: \textit{Docent beheerder} \\
Omschrijving: Zie docent en docent beheerder 

\item[C.7] Vakken onderverdelen in programmas \\
Status: \textit{Voltooid} \\
Categorie: \textit{noodzakelijke invoerfunctionaliteit} \\
Gebruikertype: \textit{Programma beheerder} \\
Omschrijving: 

\item[C.8] Vakken inladen uit bestand \\
Status: \textit{Niet voltooid} \\
Categorie: \textit{extra invoerfunctionaliteit} \\
Gebruikertype: \textit{Programma beheerder} \\
Omschrijving: Via een nader te bepalen bestand type vakken inladen in de database om gegevens uit ander database makkelijk te kunnen importeren. 
\end{itemize}

\subsubsection{Beheren van programmas}

\begin{itemize}
\item[D.1] Programmas aanmaken \\
Status: \textit{Voltooid} \\
Categorie: \textit{noodzakelijke invoerfunctionaliteit} \\
Gebruikertype: \textit{Programma beheerder} \\
Omschrijving: Programma�s worden gebruikt om vakken te bundelen 

\item[D.2] Programmas inladen uit bestand \\
Status: \textit{Voltooid} \\
Categorie: \textit{extra  invoerfunctionaliteit} \\
Gebruikertype: \textit{Programma beheerder} \\
Omschrijving: Via en nader te bepalen bestand type  programmas inladen in de database om gegevens uit ander database makkelijk te kunnen importeren.

\item[D.3] Programmas verwijderen \\
Status: \textit{Voltooid} \\
Categorie: \textit{noodzakelijke invoerfunctionaliteit} \\
Gebruikertype: \textit{Programma beheerder }\\
Omschrijving: \\[-3mm]

\item[D.4] Programmas wijzigen \\
Status: \textit{Voltooid} \\
Categorie: \textit{mogelijke invoerfunctionaliteit} \\
Gebruikertype: \textit{Programma beheerder} \\
Omschrijving: \\[-3mm]

\item[D.5] Programmas koppelen aan een student \\
Status: \textit{Voltooid} \\
Categorie: \textit{noodzakelijke/mogelijk  invoerfunctionaliteit} \\
Gebruikertype: \textit{Programma beheerder, student} \\
Omschrijving: Zie Programma beheerder 

\item[D.6] Programmas onderverdelen per faculteit \\
Status: \textit{Voltooid} \\
Categorie: \textit{mogelijke invoerfunctionaliteit} \\
Gebruikertype: \textit{Programma beheerder} \\
Omschrijving:
\end{itemize}

\subsubsection{Beheren van faculteiten}

\begin{itemize}
\item[E.1] Faculteiten aanmaken \\
Status: \textit{Voltooid} \\
Categorie: \textit{mogelijke invoerfunctionaliteit }\\
Gebruikertype: \textit{Programma beheerder} \\
Omschrijving: Faculteiten worden gebruikt om programmas in te bundelen 

\item[E.2] Faculteiten inladen uit bestand  \\
Status: \textit{Niet voltooid} \\
Categorie: \textit{mogelijke invoerfunctionaliteit} \\
Gebruikertype: \textit{Programma beheerder} \\
Omschrijving: Via een nader te bepalen bestand type faculteiten inladen in de database om gegevens uit ander database makkelijk te kunnen importeren. 

\item[E.3] Faculteiten verwijderen \\
Status: \textit{Voltooid} \\
Categorie: \textit{mogelijke invoerfunctionaliteit} \\
Gebruikertype: \textit{Programma beheerder }\\
Omschrijving: Gebruiker kan kiezen of programmas en vakken van een faculteit mee worden verwijderd 

\item[E.4] Faculteiten wijzigen \\
Status: \textit{Voltooid} \\
Categorie: \textit{mogelijke invoerfunctionaliteit} \\
Gebruikertype: \textit{Programma beheerder }\\
Omschrijving: Bepalen welke programmas tot welke faculteit behoren 
\end{itemize}

\subsubsection{Beheren van faciliteiten}

\begin{itemize}
\item[F.1] Faciliteiten aanmaken \\
Status: \textit{Deels voltooid} \\
Categorie: \textit{mogelijke invoerfunctionaliteit} \\
Gebruikertype: \textit{Faciliteiten beheerder} \\
Omschrijving: Faciliteiten omvatten lesbenodigdheden zoals projector,computer, etc. Deze kunnen enkel worden gewijzigd via de configuratie bestanden niet via de site

\item[F.3] Faciliteiten inladen uit bestand \\
Status: \textit{Voltooid} \\
Categorie: \textit{extra  invoerfunctionaliteit} \\
Gebruikertype: \textit{Faciliteiten beheerder} \\
Omschrijving: Via een nader te bepalen bestand type faciliteiten inladen in de database om gegevens uit ander database makkelijk te kunnen importeren. 

\item[F.4] Faciliteiten verwijderen \\
Status: \textit{Deels voltooid} \\
Categorie: \textit{mogelijke invoerfunctionaliteit} \\
Gebruikertype: \textit{Faciliteiten beheerder}  \\
Omschrijving: Als een faciliteit wordt verwijderd moeten alle afhankelijke beperkingen ook verwijderd. Deze kunnen enkel worden verwijderd via de configuratie bestanden niet via de site

\item[F.5] Faciliteiten wijzigen  \\
Status: \textit{Deels voltooid} \\
Categorie: \textit{mogelijke invoerfunctionaliteit} \\
Gebruikertype: \textit{Faciliteiten  beheerder} \\
Omschrijving: Deze kunnen enkel worden gewijzigd via de configuratie bestanden niet via de site.
\end{itemize}

\subsubsection{Beheren van gebouwen en lokalen}

\begin{itemize}
\item[L.1] Gebouw aanmaken \\
Status: \textit{Voltooid} \\
Categorie: \textit{noodzakelijke invoerfunctionaliteit} \\
Gebruikertype: \textit{Faciliteiten beheerder} \\
Omschrijving: Een gebouw is een verzameling van leslokalen

\item[L.2] Lokaal aanmaken \\
Status: \textit{Voltooid} \\
Categorie: \textit{noodzakelijke invoerfunctionaliteit} \\
Gebruikertype: \textit{Faciliteiten beheerder} \\
Omschrijving: Een een lokaal moet steeds aan een gebouw worden toegekend.

\item[L.3] Faciliteiten toekennen aan een lokaal \\
Status: \textit{Voltooid} \\
Categorie: \textit{mogelijke invoerfunctionaliteit} \\
Gebruikertype: \textit{Faciliteiten beheerder} \\
Omschrijving: Verschillende faciliteiten kunnen aan een lokaal worden toegekend. Dit is belangerijk om na te gaan of een lokaal voldoet aan de beperkingen opgelegd voor een vak

\item[L.4] Gebouwen inladen uit bestand \\
Status: \textit{Niet voltooid} \\
Categorie: \textit{extra invoerfunctionaliteit} \\
Gebruikertype: \textit{Faciliteiten beheerder} \\
Omschrijving: Via een nader te bepalen bestand type faciliteiten inladen in de database om gegevens uit ander database makkelijk te kunnen importeren. 

\item[L.5] Lokaal wijzigen \\
Status: \textit{In ontwikkeling} \\
Categorie: \textit{mogelijke invoerfunctionaliteit} \\
Gebruikertype: \textit{Faciliteiten beheerder}  \\
Omschrijving: Als een faciliteit wordt verwijderd moeten alle afhankelijke beperkingen ook verwijderd 

\item[L.6] Gebouw wijzigen \\
Status: \textit{Voltooid} \\
Categorie: \textit{mogelijke invoerfunctionaliteit} \\
Gebruikertype: \textit{Faciliteiten  beheerder} \\
Omschrijving: 

\item[L.7] Lokaal verwijderen \\
Status: \textit{Voltooid} \\
Categorie: \textit{noodzakelijke invoerfunctionaliteit} \\
Gebruikertype: \textit{Faciliteiten beheerder}  \\
Omschrijving: Als een faciliteit wordt verwijderd moeten alle afhankelijke beperkingen ook verwijderd 

\item[L.8] Gebouw verwijderen  \\
Status: \textit{Voltooid} \\
Categorie: \textit{noodzakelijke invoerfunctionaliteit} \\
Gebruikertype: \textit{Faciliteiten  beheerder} \\
Omschrijving: 
\end{itemize}

\subsubsection{Beheren van accounts}

\begin{itemize}
\item[G.1] Gebruikers aanmaken \\
Status: \textit{Voltooid} \\
Categorie: \textit{noodzakelijke invoerfunctionaliteit} \\
Gebruikertype: \textit{Account beheerder} \\
Omschrijving: 

\item[G.2] Gebruikers inladen uit bestand \\
Status: \textit{niet voltooid} \\
Categorie: \textit{extra invoerfunctionaliteit} \\
Gebruikertype: \textit{Account beheerder} \\
Omschrijving: Via een nader te bepalen bestand type gebruikers inladen in de database om gegevens uit ander database makkelijk te kunnen importeren. 

\item[G.3] Gebruikers verwijderen \\
Status: \textit{Voltooid} \\
Categorie: \textit{noodzakelijke invoerfunctionaliteit} \\
Gebruikertype: \textit{Account beheerder} \\
Omschrijving: Alle gegevens van de gebruiker worden uit de database verwijderd 

\item[G.4] Gebruikers wijzigen \\
Status: \textit{Voltooid} \\
Categorie: \textit{mogelijke invoerfunctionaliteit} \\
Gebruikertype: \textit{Account beheerder} \\
Omschrijving:

\item[G.5] Gebruikers blokkeren \\
Status: \textit{Niet voltooid} \\
Categorie: \textit{mogelijke  invoerfunctionaliteit} \\
Gebruikertype: \textit{Account beheerder}\\
Omschrijving: Een geblokkeerde gebruiker krijgt geen toegang tot het systeem tijdens het aanmelden 

\item[G.6] Gebruikertypes aanmaken \\
Status: \textit{Deels voltooid} \\
Categorie: \textit{mogelijke invoerfunctionaliteit} \\
Gebruikertype: \textit{Account beheerder} \\
Omschrijving: Laat toe functionaliteit te bundelen in gebruikertypes op maat. Enkel via configuration files niet via site.

\item[G.7] Gebruikertypes verwijderen \\
Status: \textit{Deels voltooid} \\
Categorie: \textit{mogelijke invoerfunctionaliteit} \\
Gebruikertype: \textit{Account beheerder} \\
Omschrijving: Sommige gebruikertypes zijn beschermd tegen verwijdering om verdere werking van de software te garanderen. Enkel via configuration files niet via site.

\item[G.8] Gebruikertypes wijzigen \\
Status: \textit{Deels voltooid} \\
Categorie: \textit{mogelijke invoerfunctionaliteit} \\
Gebruikertype: \textit{Account beheerder} \\
Omschrijving: Sommige gebruikertypes zijn beschermd tegen aanpassingen om verdere werking van de software te garanderen. Enkel via configuration files niet via site.
\end{itemize}

\subsubsection{Beheren van beperkingen}

\begin{itemize} 
\item[H.1] Tijdsbeperking aanmaken  \\
Status: \textit{Voltooid} \\
Categorie: \textit{noodzakelijke  invoerfunctionaliteit} \\
Gebruikertype: \textit{Rooster beheerder, docent} \\
Omschrijving: Een tijdsbeperking legt op waneer een vak kan worden geplant. Een docent die bijvoorbeeld een bepaalde dag in de week niet beschikbaar is, moeten dit kunnen meegeven. Algmene beperkingen, bijvoorbeeld: op welke dagen van de week er les kan worden gegeven, worden door de  rooster beheerder ingevoerd.

\item[H.2] Tijdsbeperking verwijderen \\
Status: \textit{Voltooid} \\
Categorie: \textit{noodzakelijke invoerfunctionaliteit} \\
Gebruikertype: \textit{Rooster beheerder, docent} \\
Omschrijving: Aangezien de omstandigheden die aanleiding geven tot een beperking kunnen veranderen, moet het mogelijk zijn tijdsbeperkingen te verwijderen. 

\item[H.3] Tijdsbeperking wijzigen \\
Status: \textit{Voltooid} \\
Categorie: \textit{noodzakelijke invoerfunctionaliteit} \\
Gebruikertype: \textit{Rooster beheerder, docent} \\
Omschrijving: Aangezien de omstandigheden die aanleiding geven tot een beperking kunnen veranderen, moet het mogelijk zijn om tijdsbeperkingen te wijzigen.

\item[H.4] Faciliteitbeperking aanmaken \\
Status: \textit{Voltooid} \\
Categorie: \textit{mogelijke invoerfunctionaliteit} \\
Gebruikertype: \textit{Rooster beheerder, docent} \\
Omschrijving: Een faciliteitbeperking legt op welke faciliteiten er nodig zijn voor een vak. Op basis hier van kan worden nagegaan welke lokalen er in aanmekringen koen voor het vak. 

\item[H.5] Faciliteitbeperking verwijderen \\
Status: \textit{Voltooid} \\
Categorie: \textit{mogelijke invoerfunctionaliteit} \\
Gebruikertype: \textit{Rooster beheerder, docent} \\
Omschrijving: Aangezien de omstandigheden die aanleiding geven tot een beperking kunnen veranderen, moet het mogelijk zijn faciliteitbeperking te verwijderen. 

\item[H.6] Faciliteitbeperking wijzigen \\
Status: \textit{Voltooid} \\
Categorie: \textit{mogelijke invoerfunctionaliteit} \\
Gebruikertype: \textit{Rooster beheerder, docent} \\
Omschrijving:Aangezien de omstandigheden die aanleiding geven tot een beperking kunnen veranderen, moet het mogelijk zijn faciliteitbeperking te wijzigen. \\[-3mm]
\end{itemize}

\subsubsection{Roosterplanner}

\begin{itemize}
\item[I.1] Configureren van roosterplanner \\
Status: \textit{Niet Voltooid} \\
Categorie: \textit{mogelijke invoerfunctionaliteit} \\
Gebruikertype: \textit{Software beheerder, Rooster beheerder} \\
Omschrijving: Laat toe de technische parameters van de roosterplanner aan te passen

\item[I.2] Starten van roosterplanner \\
Status: \textit{In ontwikkeling} \\
Categorie: \textit{mogelijke invoerfunctionaliteit} \\
Gebruikertype: \textit{Rooster beheerder} \\
Omschrijving: Data noodzakelijk voor de planning mag na het starten niet meer worden aangepast. Het starten van de roosterplanner bevriest dus sommige gegevens tot zijn planningstaak is gestopt.

\item[I.3] Stoppen van roosterplanner \\
Status: \textit{In ontwikkeling} \\
Categorie: \textit{mogelijke invoerfunctionaliteit} \\
Gebruikertype: \textit{Rooster beheerder} \\
Omschrijving: Het handmatig stoppen van de roosterplanner. De planner kan later weer worden gestart om zijn taak verder te zetten. 

\item[I.4] Status van de roosterplanner opvragen \\
Status: \textit{In ontwikkeling} \\
Categorie: \textit{mogelijke uitvoerfunctionaliteit} \\
Gebruikertype: \textit{Rooster beheerder} \\
Omschrijving: Het handmatig opvragen van de status van de roosterplanner. 

\item[I.5] Melden van conflicten in het rooster \\
Status: \textit{Niet voltooid} \\
Categorie: \textit{mogelijke verwerkingsfunctionaliteit} \\
Gebruikertype: \textit{Rooster beheerder} \\
Omschrijving: Als de rooster planner geen rooster kan vinden dat voldoet meldt hij dit als een conflict 
\end{itemize}

\item[I.6] Hard constraints voorzien \\
Status: \textit{Niet voltooid} \\
Categorie: \textit{noodzakelijke verwerkingsfunctionaliteit} \\
Gebruikertype: \textit{Rooster beheerder} \\
Omschrijving: Hard constraint drukken beperkingen uit die zeker moeten zijn voldaan. bv: feestdagen, geen dubbelboeking, binnen de werkuren,...
\end{itemize}

\item[I.7] Soft constraints voorzien \\
Status: \textit{Niet voltooid} \\
Categorie: \textit{noodzakelijke verwerkingsfunctionaliteit} \\
Gebruikertype: \textit{Rooster beheerder} \\
Omschrijving: Soft constraints drukken voorkeuren uit van plaats en tijd waarmee het algoritme rekening kan houden. Bv: men verkiest dat de les in een bepaald gebouw wordt gegeven of de oefeningen pas geven na de theorie. Het voldoen aan de soft constraint is echter niet noodzakelijk voor een geldige planning. 
\end{itemize}

\subsubsection{Beheren van lessenroosters}
\begin{itemize}
\item[J.1] Lessenroosters berekenen \\
Status: \textit{In ontwikkeling} \\
Categorie: \textit{mogelijke verwerkingsfunctionaliteit} \\
Gebruikertype: \textit{Roosterbeheerder} \\
Omschrijving: De rooster planner berekend automatisch roosters die voldoen aan de opgegeven beperkingen. 

\item[J.2] Lessenroosters aanmaken \\
Status: \textit{In ontwikkeling} \\
Categorie: \textit{mogelijke invoerfunctionaliteit} \\
Gebruikertype: \textit{Rooster beheerder} \\
Omschrijving: 

\item[J.3] Lessenroosters verwijderen \\
Status: \textit{In ontwikkeling} \\
Categorie: \textit{mogelijke invoerfunctionaliteit} \\
Gebruikertype: \textit{Rooster beheerder }\\
Omschrijving: 

\item[J.4] Lessenroosters wijzigen \\
Status: \textit{Voldtooid} \\
Categorie: \textit{mogelijke invoerfunctionaliteit} \\
Gebruikertype: \textit{Rooster beheerder} \\
Omschrijving: Lessen roosters kunnen handmatig worden aangespast 
\end{itemize}

%\subsubsection{Beheren van logboek }
%
%\begin{itemize}
%\item[K.1] Logboek aanmaken \\
%Status: \textit{Niet in ontwikkeling} \\
%Categorie: \textit{extra  invoerfunctionaliteit} \\
%Gebruikertype: \textit{Software beheerder} \\
%Omschrijving: Als een logboek is aangemaakt zullen belangrijke gebeurtenissen in het systeem worden bijgehouden 
%
%\item[K.2] Logboek bekijken \\
%Status: \textit{Niet in ontwikkeling} \\
%Categorie: \textit{extra uitvoerfunctionaliteit} \\
%Gebruikertype: \textit{Software beheerder} \\
%Omschrijving: 
%
%\item[K.3] Gebeurtenissen registreren \\
%Status: \textit{Niet in ontwikkeling} \\
%Categorie: \textit{extra  verwerkingsfunctionaliteit} \\
%Gebruikertype: \textit{Software beheerder} \\
%Omschrijving: 
%
%\item[K.4] Gebeurtenissen ongedaan maken \\
%Status: \textit{Niet in ontwikkeling} \\
%Categorie: \textit{extra invoerfunctionaliteit} \\
%Gebruikertype: \textit{Software beheerder} \\
%Omschrijving: 
%
%\end{itemize}

\subsubsection{Overige}

\begin{itemize}
\item[X.1] Aanmaken van de software databasestructuur via SQL \\
Status: \textit{Voltooid} \\
Categorie: \textit{noodzakelijke verwerkingsfunctionaliteit} \\
Gebruikertype: \textit{Software beheerder} \\
Omschrijving: De software brengt zelf een datastructuur aan die nodig is voor het opslaan van de gegevens  

\item[X.2] Aanpassen van de look en feel van de website \\
Status: \textit{Niet voltooid} \\
Categorie: \textit{extra invoerfunctionaliteit} \\
Gebruikertype: \textit{Software beheerder} \\
Omschrijving: Dit wordt mogelijk gemaakt door et toegankelijk maken van de opmaak bestanden.
\end{itemize}

%ZJEF

\subsection{Performantie}

Er zijn geen specifieke vereisten in verband met de snelheid van de software. Het is echter wel duidelijk uit de aard van het project, dat er een groot aantal gebruikers tegelijk de webinterface (en met gevolg de databases) moet kunnen consulteren.

%\subsection{Software system attributes}

% \subsubsection{Reliability}
% This should specify the factors required to establish the required reliability of the software system at time of delivery.


% \subsubsection{Availability}
% This should specify the factors required to guarantee a defined availability level for the entire system such as checkpoint, recovery, and restart.


\subsection{Beveiliging}
%This should specify the factors that protect the software from accidental or malicious access, use, modification, destruction, or disclosure. Specific requirements in this area could include the need to
% a) Utilize certain cryptographical techniques;
% b) Keep specific log or history data sets;
% c) Assign certain functions to different modules;
% d) Restrict communications between some areas of the program;
% e) Check data integrity for critical variables.

%Als de gebruiker zich aanmeldt, wordt een code gegenereerd voor die gebruiker (random nummer) die lokaal bijgehouden wordt door de gebruiker.\\
%Bij elk commando van de gebruiker wordt de code meegegeven.\\
%Op de server wordt een tijdelijke lijst bijgehouden die de code aan accounts verbindt. (en inlogtijd,... om ze na een tijd weg te kunnen smijten als de gebruiker niet letterlijk uitlogt). \\
%Bij elk commando van de gebruiker wordt de code in de lijst opgezocht, de rechten van de bijhorende gebruiker gecontroleerd en al dan niet kan het commando uitgevoerd worden.\\[5mm]
%Logboek van aanpassingen door Managers wordt bijgehouden.\\

\paragraph{Gebruikerristricties}

Verschillende gebruikers krijgen verschillende rechten toegewezen. Deze bepalen
tot welke informatie en tools hij toegang krijgt. Deze rechten zijn gekoppeld
aan de account van deze gebruiker. Na het inloggen zullen enkel de informatie
en tools waarvoor de gebruiker gemachtigd is, getoond worden. Om verdere
beveiliging te verzekeren, wordt ook de communicatie met de server beveiligd.
Bij het inloggen krijgt de webinterface van de gebruiker een access code toegestuurd,
op dat moment gegenereerd door de server. Deze houdt bij welke rechten bij
deze code horen. De webinterface stuurt de verkregen code mee door met elke instructie naar de server, waarop deze kan controleren of de gebruiker gemachtigd
is om de desbetreffende instructie uit te voeren.
\begin{itemize}
	\item[] \textbf{M.1 Gebruiksrechten vastleggen per gebruikertype} \\
		Status: \textit{Voltooid} \\
		Categorie: \textit{noodzakelijke invoerfunctionaliteit} \\
		Gebruikertype: \textit{Account beheerder} \\
		Omschrijving: In een file zal onder de vorm van een xml formaat worden bijgehouden worden wat de rechten zijn van elk gebruikerstype . Eventueel kan deze file via een webinterface op een gebruiksvriendelijk wijze worden aan gepast.
	
	\item[] \textbf{M.2 Gebruikersessie volgen} \\
		Status: \textit{Voltooid} \\
		Categorie: \textit{noodzakelijke verwerkingsfunctionaliteit} \\
		Gebruikertype: \textit{n.v.t.} \\
		Omschrijving: Eenmaal een gebruiker zich heeft aangemeld via een gebruikersnaam en wachtwoord zal hem een sessie nummer worden toegekend. Dit nummer laat toe de gebruiker te volgen tijdens zijn bezoek op de site en na te gaan over welke rechten de gebruiker beschikt. Om de sessie van de gebruiker tijdens het verkennen van de site bij te houden zal gebruik worden gemaakt van url-rewriting. Voor meer informatie over de implementatie van deze techniek wordt verwezen naar het SDD.

	\item[] \textbf{M.3 Gebruikersessie be�indigen na time out} \\
		Status: \textit{Voltooid} \\
		Categorie: \textit{mogelijke verwerkingsfunctionaliteit} \\
		Gebruikertype: \textit{n.v.t.} \\
		Omschrijving: Na het verstrijken van vijf minuten van inactiviteit zal het sessienummer van de gebruiker niet langer geldig zijn en wordt de sessie van de gebruiker gestopt. De gebruiker zal zich opnieuw moeten aanmelden om een nieuwe sessie te starten.
		
	\item[] \textbf{M.4 Gebruiksrechten controleren} \\
		Status: \textit{Voltooid} \\
		Categorie: \textit{noodzakelijke verwerkingsfunctionaliteit} \\
		Gebruikertype: \textit{n.v.t.} \\
		Omschrijving: Als een gebruiker een bepaalde functionaliteit aan de server opvraagt zal hierbij steeds zijn sessie nummer worden  meegedeeld, wat mogelijk maakt na te gaan of de gebruiker hier toe gemachtigd is. 

	\item[] \textbf{M.5 Webpagina op maat van rechten} \\
		Status: \textit{Voltooid } \\
		Categorie: \textit{noodzakelijke verwerkingsfunctionaliteit} \\
		Gebruikertype: \textit{n.v.t.} \\
		Omschrijving: De gebruiker zal op de website enkel de mogelijkheden voorgeschoteld krijgen die in overeenkomst zijn met zijn rechten.
		
\end{itemize}
		

\paragraph{Data integriteit}

Enerzijds zal er de mogelijkheid geleverd worden, aan de daarvoor gemachtigde gebruiker(s), om op de server een back-up van de databases (zoals accounts,
leslokalen, vakken,...) te maken en desnoods een rollback uit te voeren. Anderzijds zal er op de server een logboek bijgehouden worden met de aanpassingen
aan de databases, die door verschillende gebruikers gemaakt kunnen worden. 

\subparagraph{Beheren van logboek}

\begin{itemize}
	\item[K.1] \textbf{Logboek aanmaken} \\
		Status: \textit{niet Voltooid} \\
		Categorie: \textit{extra invoerfunctionaliteit} \\
		Gebruikertype: \textit{Software beheerder} \\
		Omschrijving: Als een logboek is aangemaakt zullen belangrijke gebeurtenissen in het systeem worden bijgehouden

	\item[K.2] \textbf{Logboek bekijken} \\
		Status: \textit{niet Voltooid} \\
		Categorie: \textit{extra uitvoerfunctionaliteit} \\
		Gebruikertype: \textit{Software beheerder} \\
		Omschrijving:

	\item[K.3] \textbf{Gebeurtenissen registreren} \\
		Status: \textit{niet Voltooid} \\
		Categorie: \textit{extra verwerkingsfunctionaliteit} \\
		Gebruikertype: \textit{Software beheerder} \\
		Omschrijving:

	\item[K.4] \textbf{Gebeurtenissen ongedaan maken} \\
		Status: \textit{niet Voltooid} \\
		Categorie: \textit{extra invoerfunctionaliteit} \\
		Gebruikertype: \textit{Software beheerder} \\
		Omschrijving: 
		
\end{itemize}

		
\subparagraph{Beheren van database}
\begin{itemize}
	\item[N.1] \textbf{Back-up maken van database} \\
		Status: \textit{Niet voltooid} \\
		Categorie: \textit{extra verwerkingsfunctionaliteit} \\
		Gebruikertype: \textit{Software beheerder} \\
		Omschrijving: 

	\item[N.2] \textbf{Database configureren} \\
		Status: \textit{Deels voltooid} \\
		Categorie: \textit{noodzakelijke/extra invoerfunctionaliteit} \\
		Gebruikertype: \textit{Software beheerder} \\
		Omschrijving: In een file met xml formaat zullen de database gegevens worden opgeslagen op de server. Eventueel kunnen deze gegevens ook via een webinterface worden ingevoerd. 
\end{itemize}


\paragraph{Account beveiliging}

Account paswoorden zouden kunnen ge�ncrypteerd verstuurd en opgeslagen worden om de veiligheid van de accounts te verhogen. In hoever dit noodzakelijk is, is nog maar de vraag, aangezien het merendeel van de accountgegevens publiek toegankelijk zijn voor gastgebruikers. 

\begin{itemize}
	\item[L.1] \textbf{Encrypteren van account gegevens tijdens communicatie} \\
		Status: \textit{niet Voltooid} \\
		Categorie: \textit{extra verwerkingsfunctionaliteit} \\
		Gebruikertype: \textit{n.v.t.} \\
		Omschrijving: 
		
	\item[L.2] \textbf{Encrypteren van account gegevens in de database} \\
		Status: \textit{niet Voltooid} \\
		Categorie: \textit{extra verwerkingsfunctionaliteit} \\
		Gebruikertype: \textit{n.v.t.} \\
		Omschrijving: 
\end{itemize}


% \subsubsection{Maintainability}
% This should specify attributes of software that relate to the ease of maintenance of the software itself. There may be some requirement for certain modularity, interfaces, complexity, etc. Requirements should not be placed here just because they are thought to be good design practices.


% \subsubsection{Portability}
% This should specify attributes of software that relate to the ease of porting the software to other host machines and/or operating systems. This may include the following:
% a) Percentage of components with host-dependent code;
% b) Percentage of code that is host dependent;
% c) Use of a proven portable language;
% d) Use of a particular compiler or language subset;
% e) Use of a particular operating system.

% \subsection{Other requirements}

%\newpage
%\section*{Index}


 \end{document}
